% Зачем: Изменение надписи для списка литературы
% Почему: Пункт 2.8.1 Требований по оформлению пояснительной записки.
% \renewcommand{\bibsection}{\sectioncentered*{Cписок использованных источников}}
% \phantomsection\pagebreak% исправляет нумерацию в документе и исправляет гиперссылки в pdf
\sectioncentered*{список использованных источников}
\addcontentsline{toc}{section}{Список использованных источников}
% \addcontentsline{toc}{section}{Cписок использованных источников}

% Зачем: Печать списка литературы. База данных литературы - файл bibliography_database.bib

1. Столлингс, У. Структурная организация и архитектура компьютерных
систем/ У. Столлингс. 5-е изд. – М.: "Вильямс", 2001. Пер. с англ. – 892 с.

2. Таненбаум, Э. Архитектура компьютерных систем/ Э. Таненбаум. 4-е
изд. – М.: "ПИТЕР", 2002. Пер. с англ. – 698 с.

3. Цилькер, Б.Я. Организация ЭВМ и систем/ Б.Я. Цилькер, С.А. Орлов. –
М.: "Питер", 200. – 668 с.

4. Грушвицкий, Р. Проектирование систем на микросхемах программируемой логики/ Р. Грушвицкий. – СПб.: "Питер", 2002. – 608 с.

5. Угрюмов Е. Цифровая схемотехника. - М.: "С-Петербург", 2001, 518 стр.

6. Майоров С.А. Введение в микроЭВМ. - Л.: Машиностроение, 1988.

7. Солонина А. Алгоритмы и процессоры цифровой обработки сигналов. -
СПб.: "Питер", 2001. – 464 с.

8. Шагурин И.И. Процессоры семейства Intel Р6. Архитектура, программирование, интерфейс. – М.: "Телеком", 2000. – 248 с.

9. Рудометов Е. Материнские платы и чипсеты. – СПб.: "Питер", 2000. –
256 с.

10. Бибило П.Н. Синтез логических схем с использованием языка VHDL.
М.: СОЛОН-Р, 2002.

11. Антонов А. П. Язык описания цифровых устройств AlteraHDL. - М. :
РадиоСофт, 2001.

12. Глецевич И.И., Прытков В.А., Отвагин А.В. Методические указания по
дипломному проектированию для студентов специальности 40 02 01 «Вычис-
лительные машины, системы и сети». – Минск БГУИР, 2009, 99 с.
% \bibliography{bibliography_database}
