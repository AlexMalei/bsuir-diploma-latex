\sectioncentered*{Заключение}
\addcontentsline{toc}{section}{Заключение}
%%
%% ВНИМАНИЕ: этот реферат не соответствует СТП-01 2013
%% пример оформления реферата смотрите здесь: http://www.bsuir.by/m/12_100229_1_91132.docx
%%

% \vspace{4\parsep}

	В заключении хочется сказать, что мы постарались сделать микро-ЭВМ настолько эффективной и оптимальной, насколько это возможно.

	Разработанная микро-ЭВМ имеет следующие характеристики:
    \begin{itemize}
        \item Разрядность АЛУ: 16 бит;
        \item Принстонскаяархитектура доступа к памяти.
        \item 4 Кбайт памяти. 2 кбайт – синхронная ROM память команд, 2 кбайт – оперативная синхронная память данных;
        \item 10 16-разрядных регистров общего назначения;
        \item Стековая память размером 22 байта (11 регистров);
        \item Конвейерная обработка команд, состоящая из двух стадий;
        \item 16 различных команд: 8 команд АЛУ, 3 команды пересылки данных, 2 команды работы со стеком, 2 команды перехода, команда завершения работы процессора;
    \end{itemize}

	Отдельно хочется сказать о возможности расширения данной системы в будущем. Вся суть проектирования состояла в том, чтобы везде, где только можно, оставлять возможности для дальнейшего расширения системы. Все блоки проектировались с целью снизить зависимость между друг другом, поэтому их интерфейсы хорошо расширяются. Удачными дополнениями к системе станут кэш, конвейер, контроллер прямого доступа в память, предсказатель переходов.

\clearpage
