{
    % \newgeometry{top=1.25cm,bottom=1.25cm,right=1cm,left=2cm,twoside}
    \setlength{\parindent}{0em}

    \newcommand{\lineunderscore}{\uline{\hspace*{\fill}}}
    \newcommand\tab[1][1cm]{\hspace*{#1}}

    \begin{center}
        Министерство образования Республики Беларусь\\[1em]
        Учреждение образования\\
        БЕЛОРУССКИЙ ГОСУДАРСТВЕННЫЙ УНИВЕРСИТЕТ \\
        ИНФОРМАТИКИ И РАДИОЭЛЕКТРОНИКИ\\[1em]
    \end{center}

    \begin{minipage}{\textwidth}
        \begin{flushleft}
            Факультет: КСиС. Кафедра: ЭВМ. \\
            Специальность: 40 02 01 ``Вычислительные машины, системы и сети''.
            Специализация: 400201-01 ``Проектирование и применение локальных компьютерных сетей''.
        \end{flushleft}
    \end{minipage}\\[1em]

    \begin{minipage}{\textwidth}
        \begin{flushright}
            \begin{tabular}{p{0.40\textwidth}}
                УТВЕРЖДАЮ \\
                Заведующий кафедрой ЭВМ \\
                \underline{\hspace*{5em}}Д.И.~Самаль \\
                <<\underline{\hspace*{4ex}}>>\underline{\hspace*{6em}}2018 г.
            \end{tabular}
        \end{flushright}
    \end{minipage}

    \begin{center}
        ЗАДАНИЕ \\
        по дипломному проекту студента \\
        Стаховского Антона Владимировича
    \end{center}

    \begin{flushleft}
        \begin{legal}
        \item Тема проекта: <<Система функционального контроля комплекса машин управления артиллерийского дивизиона>> --
            утверждена приказом по университету от 25 января 2018 г. \No{177}-с

            \vspace{1em}

        \item Срок сдачи студентом законченного проекта: 1 июня 2018 г.

            \vspace{1em}

        \item Исходные данные к проекту:

            \begin{legal}
            \item Операционная система: CentOS 6.
            \item Языки программирования: C++.
            \item Фреймворки: Qt 4.8
            \item Система управления базами данных: PostgreSql
            \item Среда разработки: Visual Studio 2010.
            \end{legal}

            \vspace{1em}

        \item Содержание пояснительной записки (перечень подлежащих разработке вопросов):
            Введение.
            1. Обзор литературы.
            2. Системное проектирование.
            3. Функциональное проектирование.
            4. Разработка программных модулей.
            5. Программа и методика испытаний.
            6. Руководство пользователя.
            7. Экономическая часть.
            Заключение.
            Список использованных источников.
            Приложения.

            \vspace{1em}

        \item Перечень графического материала (с точным указанием обязательных чертежей):
            \begin{legal}
            \item Вводный плакат. Плакат
            \item Система функционального контроля технических средств КМУ артиллерийского дивизиона. Схема структурная.
            \item Система функционального контроля технических средств КМУ артиллерийского дивизиона. Схема программы.
            \item Система функционального контроля технических средств КМУ артиллерийского дивизиона. Диаграмма последовательности.
            \item Система функционального контроля технических средств КМУ артиллерийского дивизиона. Диаграмма классов.
            \item Система функционального контроля технических средств КМУ артиллерийского дивизиона. Схема адресации.
            \end{legal}

            \vspace{1em}

        \item Содержание задания по экономической части: ``Технико-экономическое обоснование разработки системы функционального контроля технических средств комплекса машин управления артиллерийского дивизиона''.

        \end{legal}

    \end{flushleft}

    ЗАДАНИЕ ВЫДАЛ \hfill{} Т.\,Л.~Слюсарь \\

    \begin{center}
        КАЛЕНДАРНЫЙ ПЛАН
    \end{center}

    \begin{table}[!htb]
        \small
        \begin{tabular}{
                | >{\centering}m{0.40\textwidth}
                | >{\centering}m{0.08\textwidth}
                | >{\centering}m{0.25\textwidth}
            | >{\centering\arraybackslash}m{0.16\textwidth}|}
            \hline Наименование этапов дипломного проекта & Объем этапа, \% & Срок выполнения этапа & Примечания \\
            \hline Подбор и изучение литературы & 10 & 25.01--01.02 & \\
            \hline Структурное проектирование & 10 & 01.02--18.03 & \\
            \hline Функциональное проектирование & 20 & 18.03--30.03 & \\
            \hline Разработка программных модулей & 30 & 30.03--22.04 & \\
            \hline Программа и методика испытаний & 10 & 22.04--04.05 & \\
            \hline Расчет экономической эффективности & 10 & 04.05--18.05 & \\
            \hline Оформление пояснительной записки & 10 & 18.05--01.06 & \\
            \hline
        \end{tabular}
    \end{table}

    Дата выдачи задания: 23 марта 2018 г.\\[1em]
    Руководитель \hfill{} Т.\,В.~Державская \\[1em]
    ЗАДАНИЕ ПРИНЯЛ К ИСПОЛНЕНИЮ \tab \uline{\hspace*{4em}}

    \clearpage

    % \restoregeometry
}
