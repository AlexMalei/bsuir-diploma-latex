\section{Анализ и оптимизация разработанной микро-ЭВМ}
Для оптимизации работы системы, улучшения масштабируемости и модульности при разработке микро-ЭВМ применялись различные приемы:

    \begin{itemize}
        \item Все команды имеют одинаковую длину, это позволяет упростить схему, тем самым снижая ее стоимость.
        \item Все операнды в команде занимают фиксированные адреса, тем самым сильно упрощается выборка команд. Добавление новых команд также не вызовет особых сложностей.
        \item Все стадии исполнения команды занимают фиксированное количество тактов -- упрощается логика устройства управления.
        \item Адреса памяти при прямой адресации располагаются в младших битах слова. Это позволяет облегчить отладку схемы.
        \item Простое строение адресного пространство памяти. Старший бит адреса определяет, какому устройству принадлежит адрес. Тем самым снижаются затраты на определение принадлежности адреса, но также это приводит к тому, что на ROM отводится половина всего адресного пространства.
    \end{itemize}

	Из недостатков системы можно указать следующие:
    \begin{itemize}
        \item Отсутствие кэша. При обращении к памяти теряется большое количество тактов. Это особенно большая проблема в Принстонской архитектуре, потому что данные и инструкции передаются через одну шину, тем самым сильно снижая производительность системы.
        \item Отсутствие контроллера прямого доступа к памяти. В случае необходимости передачи данных между каким-нибудь внешним источником и памятью необходимо затрачивать процессорное время.
        \item Отсутствует конвейеризация. Наличие конвейера позволило бы уменьшить среднее время выполнения одной команды.
        \item Отсутствие арбитража шин. Нету единой схемы доступа к устройствам, в результате чего происходит дублирование компонентов и увеличение сложности схемы.
    \end{itemize}

