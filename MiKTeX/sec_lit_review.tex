\section{Обзор литературы}
\label{sec:lit_review}

\subsection{Обзор существующих аналогов}
\label{sub:lit_review:analogues}

\subsection{Аналитический обзор}
\label{sub:lit_review:analitics}
Технические средства(ТС) --  изделия, оборудование, аппаратура и их составные части, функционирующие на основании законов электротехники, радиотехники и электроники и содержащие электронные компоненты и схемы.

КМУ -- комплекс машин управления.
В одном дивизионе имеется несколько машин разного уровня управления, содержащих в своем составе разные ТС.
Например, метеокомплект стоит только на нескольких машинах, радиостанции имеются в каждой машине, бесплатформенная инерциальная навигационная система(БИНС) присутствует также на каждой машине, но среди машин КМУ типы устройства БИНС отличаются от машины к машине, локальная вычислительная сеть(ЛВС) присутствует в каждой машине.
Программное обеспечение написано для всех машин КМУ с возможностью выборки подключенных ТС.
Доступный фукнционал также может отличаться от компьютера к компьютеру в пределах машины из-за строго разграничения по уровню доступа пользователя.

Разработанное в ходе дипломного проектирования программное обеспечение предназначено для развертывания в подвижном комплексе средств автоматизации управления.
Подробное описание струтуры данного комплекса приведено в ~\cite{patent_2263960}.
Этот подвижный комплекс средств автоматизации управления, размещенный в подвижном объекте на шасси автомобиля повышенной грузоподъемности, содержит четыре автоматизированных рабочих места(АРМ) должностных лиц, размещенных в кузове-фургоне подвижного объекта, оборудованных средствами вычислительной техники и средствами передачи данных, одно АРМ оператора на базе портативного компьютера типа Notebook, два выносных АРМ на базе портативного компьютера типа Notebook, радиорелейную станцию с антеннами, коротковолновую (KB) радиостанцию, две ультракоротковолновые (УКВ) радиостанции, аппаратуру каналообразования, локальную вычислительную сеть (ЛВС), мультиплексор, телефонный коммутатор, четыре телефонных аппарата оперативной связи системы МБ, один телефонный аппарат системы МБ для технологической связи, радиоприемник системы точного времени, матричный принтер, лазерный принтер, УКВ радиостанцию, установленную в кабине водителя для обеспечения радиосвязи при движении в колонне.

Программа функционального контроля предназначена для осуществления автоматизации процессов проведения тестирования
технических\break средств.
Программа функционального контроля обеспечивает выполнение следующих функций:
\begin{legal}
    \item тестирование средств автоматизации, локальной вычислительной сети(ЛВС), визуализацию информации о доступных в ЛВС автоматизированных рабочих местах(АРМ);
    \item тестирование и настройку средств связи;
    \item тестирование и настройку средств измерения.
\end{legal}
